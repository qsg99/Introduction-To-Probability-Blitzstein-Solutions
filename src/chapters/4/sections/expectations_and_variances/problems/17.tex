\begin{enumerate}[label=(\alph*)]
\item The expected number of children in a randomly selected family during a
particular era is $\text{E}(X) = \sum_{k=0}^{\infty}k\frac{n_{k}}{\sum_{k=0}^
{\infty}n_{k}} = \frac{m_{1}}{m_{0}}$.

\item The expected number of children in the family of a randomly selected child
is $\text{E}(X) = \sum_{k=0}^{\infty}k\frac{kn_{k}}{\sum_{k=0}^{\infty}kn_{k}} =
\frac{m_{2}}{m_{1}}$.

\item answer in part $b$ is larger than the answer in part $a$. Since the
average in part $a$ is taken over randomly selected families, families with
fewer children are weighted the same as families with more children. The average
in part $b$, on the other hand, is taken over individual children, skewing the
weights in favor of families with more children.

To see how this relates to the Wrigley-Field problem: we see that it is possible for one era to have a higher random family size, while having a lower size of a random child's family. Consider the following simple example: Let Era 1 have 4 families with 3 children, and 4 families with 0 children. Let Era 2 have 7 families with 2 children, and 1 family with 1 child. Era 2 has a higher number of children per family overall, but it is clear the children from Era 1 are always from a family with more children than Era 2. 
\end{enumerate}
